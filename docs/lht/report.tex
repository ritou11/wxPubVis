\documentclass[a4paper,12pt]{article}
\usepackage[notoc,noabs]{HaotianReport}

\title{期末大作业个人报告}
\author{刘昊天}
\authorinfo{电博181班, 2018310648}
\runninghead{清华大学《数据可视化》2019春季学期}
\studytime{2019年6月}

\begin{document}
    \begin{abstract}
    TODO
        随着计算机技术、网络技术、控制技术及人工智能等的飞跃发展,智能化社会已成为新世纪的发展趋势.在此之下,智能家居也随之迅猛发展起来。最早的体现出智能家居概念的建筑出现在 1984 年的美国,随后世界各国纷纷涌现出各种智能家居解决方案,目前已在发达国家有一定的规模。多年来,智能家居经历了从 “住宅自动化” 到 “住宅智能化” 的过渡,成为了当今世界最热门的技术领域之一.20 世纪 90 年代,我国的科技水平飞速发展,国际上流行的智能化生活方式也逐渐进入我们的视野中,智能家居首先出现在发展较为迅速的沿海城市,随着经济的发展,智能化的家居建筑正快速的向内地市场延伸.随着改革开放的深入,中国的经济飞速发展,人们对于生活的舒适度和体验感的要求越来越高,各种电子技术也大量应用于人们的日常生活中,计算机网络的覆盖程度越来越广泛,人们可以随时随地的连接到网络中,甚至生活中的各种设备都可以连接到网络,给人们带来完全不同的生活方式,因此,家居智能化必然成为一种趋势。
        \begin{keywords}
            健身,智能硬件
        \end{keywords}
    \end{abstract}
    \maketitle
    %\newpage
    \section{第一段} % (fold)
    \label{sec:第一段}
    这里是第一段内容。本产品是基于Ti SImplelink CC3200模块的智能计量插座。用户可以在我们的网站https://spp.nogeek.top上注册一个账号,将任意个智能插座产品绑定在自己的账号下。随后,可以通过网站上的控制台,查看各个插座所带负载的电能使用情况,并控制插座的开关情况。通过这种方式,用户可以实现对家中指定负荷的随时监控,改善用电情况,优化用电结构;同时可以远程操控负载的通断,随时停止忘记关闭的设备,或在到家前提前打开设备。
    % section 第一段 (end)
    \label{applastpage}
    \newpage
    \bibliography{report}
    \bibliographystyle{unsrt}
\iffalse
\begin{itemize}[noitemsep,topsep=0pt]
%no white space
\end{itemize}
\begin{enumerate}[label=\Roman{*}.,noitemsep,topsep=0pt]
%use upper case roman
\end{enumerate}
\begin{multicols}{2}
%two columns
\end{multicols}
\fi
\end{document}
