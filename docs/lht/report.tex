\documentclass[a4paper,12pt]{article}
\usepackage[notoc,noabs]{HaotianReport}

\title{期末大作业个人报告}
\author{刘昊天}
\authorinfo{电博181班, 2018310648}
\runninghead{清华大学《数据可视化》2019春季学期}
\studytime{2019年6月}

\begin{document}
    \maketitle
    \section{个人在项目中的工作}
      \begin{enumerate}
        \item 完成开题、选题、论述背景、形成团队、组织讨论等事务性工作。
        \item 完成了代码主体的开发,包括选定框架、编写前端后端代码、改写开源爬虫、调试、部署等。
        \item 整合、调试并改良其他成员的代码,使其能够正常在项目中使用。
        \item 完成了最终实现的5个可视化任务中4个的可视化工作。
        \item 参与了多次小组讨论,并贡献了部分算法的实现思路。
        \item 参与制作结题答辩PPT。
      \end{enumerate}
    \section{遇到的困难}
    \begin{enumerate}
      \item 项目思路不够明确。在立项之初,我们仅仅有一个基本的想法和出发点,本人也制定了粗糙的实现方向。然而,在多人协作时,思路的明确至关重要。项目路线不清晰、开发协作参与热情不高,这是我们在初期遇到的主要问题。为了应对这个问题,我们增加了多次碰面,并制作了思维导图。
      \item 框架上手难度大,项目成员难以短时间熟悉。本次选用了React+D3的前端实现方案,以React作为前端交互开发的框架,以D3作为可视化工具。这种模式要求开发者掌握React的用法,并能熟练地开发调试,而本组成员基本不掌握这方面的知识。另外,后端采用Nodejs+MongoDB作为实现方案,采用GraphQL作为接口规则,这些工具本组成员也不熟悉。该问题的解决方式是,小组成员分别开发,最后本人进行改写、合并、调试与改良。这也消耗了大部分的开发时间。
    \end{enumerate}
    \section{心得体会}
    我们组的题目为《微信公众号数据可视化》。该项目选题之初,是出于本人的心血来潮,主要是因为对新媒体的浓厚兴趣,以及偶然看到了微信公众号推送内容的爬取方法。在开始组队后,也曾想过放弃本题目加入另一组,但在看过所有其他选题过后,还是觉得这个题目是自己最感兴趣的,因此开始了组队作业之旅。

    这里提一个课程建议,最好能够建立课程微信群,让大家方便联系,更好地组队。本次组队的手段非常单一,只能在没什么人看的学堂讨论区发布帖子,有点类似姜太公钓鱼,整个过程有点麻烦。

    在作业的开发过程中,最大的问题是大家对这个题目的认识,以及相关的背景能力不是很充足。该题目涉及到NLP应用层面的知识,而即便提出题目的本人也只是略微了解,因此在前期规划时难度很大,只能走一步看一步。最终的大部分可视化任务,都是在顾玥同学完成了NLP相关计算后,本人未经讨论直接做出的,这不是一个很科学的合作模式。

    在可视化方面,本次作业让我认识到了可视化的难度。一个合格的可视化程序,是要全面地考虑数据的各种情况的。在开发中,还往往面临着部分数据缺失时的处理。

    本课程的最大收获,就是深切地体会了完成一个可视化任务的各个阶段,其本质就是数据的映射。我们的任务,就是将在网络上获取的文本数据,以及相关的结构化数据,通过NLP映射到各种词汇、主题的统计数据,再重新组织清洗,映射到几何数据,最终通过像素呈现出来。这种思路层面的收获,会让我的后续科研工作受益匪浅。

    感谢老师一学期以来的精彩授课!感谢助教在开题时提供的建议,以及平时作业的认真负责,特别感谢助教在最后验收时为我们的特殊情况改变验收时间,特意跑到实验室!
    \label{applastpage}
    % \newpage
    % \bibliography{report}
    % \bibliographystyle{unsrt}
\iffalse
\begin{itemize}[noitemsep,topsep=0pt]
%no white space
\end{itemize}
\begin{enumerate}[label=\Roman{*}.,noitemsep,topsep=0pt]
%use upper case roman
\end{enumerate}
\begin{multicols}{2}
%two columns
\end{multicols}
\fi
\end{document}
